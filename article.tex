\documentclass[12pt]{article}
\usepackage[utf8]{inputenc}
\usepackage[T1]{fontenc}
\usepackage[ngerman]{babel}
\usepackage{amsmath}
\usepackage{amssymb}
\usepackage{hyperref}

\usepackage{tikz-cd}
\usetikzlibrary{quotes,babel,angles}


\newcommand{\R}{\mathbb{R}}
\newcommand{\Set}{\mathbf{Set}}
\newcommand{\Vect}{\mathbf{Vect}}

\title{Can free vector space functor be the right-adjoint of a forgetful functor?}
\author{Grzegorz Milka}
\date{27. December 2020}

\begin{document}

\maketitle
\tableofcontents

\section{Intro}

I wanted to better understand adjunctions, so I asked myself whether I know any
example where \(L \dashv R\) but not \(R \dashv L\).

I knew that free and forgetful functors
were classical examples of adjunct functors. If I flipped the relation,
could a forgetful functor be a left-adjoint of a free functor? Say, a free
functor of linear spaces, \(F: \Set \rightarrow \Vect_{\R}\)?

\section{Understanding the objects}

Let's start with exploring the relevant objects.

\subsection{$F$}

Let $F: \Set \rightarrow \Vect_{\R}$ be the free functor.

For any $B \in \Set$, $FB = lin(b), b \in B$ that is $FB$ is the vector space over $\R$
with elements of $B$ as its base.

How does $F$ map functions? Let $f: B \rightarrow C, f \in hom(Set)$, then

 \begin{align*}
  Ff: FB &\rightarrow FC\\
  Ff\left(\sum k_ib_i\right) &= \sum k_i f(b_i)
  \end{align*}

\subsection{$U$}

Let $U: \Vect_{\R} \rightarrow \Set$ be the forgetful functor.

For any $V \in \Vect$, $UB = B$. To avoid confusion, I'll note elements of $B
\in \Set$ with a hat, e.g. $\hat{b}$.

Functions are also mapped 1-to-1.

\subsection{$FU$}

$FU: \Vect \rightarrow \Vect$ is an endofunctor. I'll denote mapped vector
spaced $V$ as $\hat{V}$.  Elements of $\hat{V}$ have form $k(\hat{lv}), v \in
V, \, k, l \in \R$.

It's crucial to understand how $FU$ maps linear morphisms. The mapped morphism are linear in the vector space introduced by $F$, any they also act on the unstructured elements under the hat, $\hat{\cdot}$.
For any $f \in Hom(\Vect)$,
\[FUf\left(\sum k_i(\widehat{\sum l_j v_j})\right)
= \sum k_i \left(
                 \widehat{f \left(\sum l_j v_j \right)}
           \right) \]

\section{$U \dashv F$}

The previous section only explored the properties of $U$ and $F$. This section now dives into $(U \dashv F)$.

% https://tikzcd.yichuanshen.de/#N4Igdg9gJgpgziAXAbVABwnAlgFyxMJZABgBoBGAXVJADcBDAGwFcYkQAlEAX1PU1z5CKcqWLU6TVuxgB9cjz4gM2PASIBmMRIYs2iEACY5C3v1VCiowzqn6QAHQdxmAWwAEAa1kArJwAt6HGAoX25Fc0F1FC0bGl1pAwAxAFUubgkYKABzeCJQADMAJwhXJFEQHAgkQ3i7dgKIkGLS8poqpA0zZpKyxFrK6sQAFjq9dicYHHpZYCdXIP8AIyXgDm5w7pa+rUGkUclx5JTGrd6kMj3ECoT7AEkoJu2L9qHdxnolmEYABQE1YQgRgwAo4EBjRIgVI8SjcIA
\begin{tikzcd}
                                    & e_1 \arrow[rr, "f"] \arrow[dd, "\eta_{\R}"]      &  & 2e_1 \arrow[d, "\eta_{\mathbb{R}}"] \\
  R \arrow[ru, "Id"] \arrow[rd, "FU"] &                                     &  &              \sum 2k_j\hat{d_j}                        \\
                                    & \sum k_j\hat{d_j} \arrow[r, "FUf"] & \sum k_j\hat{2d_j}  &
\end{tikzcd}

Let's assume a natural transformation $\eta: Id_V \rightarrow FU$ exists. Let's
take $V = \R$ with $e_1$ as its base.
Let's denote the component of $\eta$ on $V$ as $\eta_V$.
By definition of natural transformation, $\eta_V$ is a linear morphism. Let's
denote mapped base as $\eta_V(e_1) = \sum_j  k_{j}\hat{d_{j}}, d_{j} \in V$.

Let's now take a linear morphism $f: V \rightarrow V, f(e) = 2e$.
By definition of natural transformation we should now have that $\eta_V \circ f
= FUf \circ \eta_V$. Let's see how this equation acts on $e_1$.

\begin{enumerate}
  \item $L = \eta_V(f(e_1)) = \eta_V(2e_1) = 2\sum_j k_{j} \hat{d_{j}}$ (the last transformation used the fact that $\eta_V$ is linear).
  \item $R = FUf(\eta_V(e_1)) = FUf\left(\sum_j k_{j} \hat{d_{j}}\right)
      = \sum_j k_{j} \widehat{2d_{j}}$
\end{enumerate}

The only way $L = R$ is if $\eta_V(e_1) = 0$. $\eta$ must be trivial: its components map every
vector space to zero.

\section{Conclusion}

I expected $U \dashv L$ to be false but instead got a result that such
adjunction may exist, but it's trivial.

I'm not sure if I missed any crucial assumption that would show impossibility,
or whether mathematicians have something interesting about this result.

\section{See Also}

* \href{https://ncatlab.org/nlab/show/Vect}{Vect on nLab}


\end{document}
